\listfiles
\documentclass{article}

\usepackage{amsmath}
\usepackage{amssymb}
\usepackage{mathtools}
\usepackage{listings}
\usepackage[]{algorithm2e}

\DeclarePairedDelimiter\floor{\lfloor}{\rfloor}
\DeclarePairedDelimiter\ceil{\lceil}{\rceil}
\DeclareMathOperator{\cl}{cl}
\DeclareMathOperator{\E}{E}
\def\Z{\mathbb{Z}}
\def\N{\mathbb{N}}
\def\R{\mathbb{R}}
\def\Q{\mathbb{Q}}
\def\K{\mathbb{K}}
\def\T{\mathbb{T}}
\def\O{\mathcal{O}}
\def\B{\mathcal{B}}
\def\XX{\mathfrak{X}}
\def\YY{\mathfrak{Y}}
\def\AA{\mathfrak{A}}
\def\ZZ{\mathfrak{Z}}
\def\BB{\mathcal{B}}
\def\UU{\mathcal{U}}
\def\MM{\mathcal{M}}
\def\M{\mathfrak{M}}
\def\l{\lambda}
\def\L{\Lambda}
\def\<{\langle}
\def\>{\rangle}

\usepackage[a4paper,margin=1in]{geometry}

\setlength{\parindent}{0cm}
\setlength{\parskip}{1em}

\title{CS 450 Assignment K0}
\date{}

\begin{document}
\maketitle


\section*{1. How To Operate Program}

The .elf artifact is committed in the git repository at [TODO]. See section 4 for a link to the code repsoitory. Alternatively see section 5 for instructions to build from source. Once the artifact is obtained, load it into redboot and run \texttt{go}.

\section*{2. Group Member Names}

Xuanji Li

Lennox Fei

\section*{3. Kernel Structure}

\subsection*{Context Switch}

Context switch from user mode to kernel is done by the \texttt{SWI <n>} instruction, and context switch from kernel to user mode is done by restoring a saved user mode CPSR register from memory which sets the processor mode to user mode.

At startup, the \texttt{setUpSWIHandler} handler function is used to install the \texttt{sys\_handler} as the handler for \texttt{SWI}. This is done by writing the absolute address of \texttt{sys\_handler} to \texttt{0x28} and the instruction \texttt{LDR pc, [pc, \#0x18]} to \texttt{0x08}. \texttt{sys\_handler} expects to be in svc mode, with the sp pointing to a trap frame with the saved kernel context. It

\begin{enumerate}
\item Get the exact \texttt{SWI} instruction and pass it to the syscall handling system
\item Get the user mode sp and write it to the current TD
\item Restore kernel context from the trap frame
\end{enumerate}


\subsection*{Task Descriptiors}

\subsection*{Syscalls}


\section*{4. Code repository}

https://git.uwaterloo.ca/f5fei/chos

\section*{5. Output and explanation}

[TODO]

\end{document}
